\documentclass[9pt,technote]{IEEEtran}
\usepackage{cite}

\title{SOK - A secure operating system architecture}
\author{Christopher Zell\\
        zell.christopher@fu-berlin.de}       


\begin{document}
\maketitle

  \begin{abstract}
    Abstract...
  \end{abstract}
  
  \section{Einleitung} \label{sec:intro}
    In folgendem Paper wird auf die Unterschiedlichen sicheren Betriebssysteme eingegangen ,
    deren Vorgehen erkl\"art und die Unterschiede aufgezeigt. Zum gr\"o\ss ten Teil
    wird auf den Kernel und die verschiedenen Zugriffsschutz Mechanismen eingegangen.
  \section{Zugriffsschutz Mechanismen} \label{sec:protection}
   \nocite{*}   
  
   \subsection{Multics}  
   
   Informationen werden in Segmente gespeichert und koennen geteilt werden sie beinhalten eigene Attribute 
   ueber deren Groesse sowie die Zugriffsrechte.
   
   In Multiplexed Information and Computing Service - Multics  liefer Segmentierung eine allgemeine Basis fuer den direkten Zugriff
   und das Teilen von online Informationen durch erfuellen zweier Desigin Ziele. : 1 direkt zugriff 2 Zugriffskontrolle
   
   Vorteil der direkten Adressierung ist das die Informationen nicht mehr kopiert werden muessen. Da alle instruktionen und Daten
   im System vom Prozessor adressierbar sind. D.h. duplizierung von Prozeduren und Daten is unnoetig soll heissen Abbilder von Prozessen muessen
   nicht vorher geladen werden die orginalen Prozeduren koennen direkt adressiert werden.
   
  Dies soll dem Programmierer eine attraktive reduktion der programm komplexitaet versprechen.
  
  In Multics Segemente sind Packete von Informationen welche direkt Adressiert und Kontrolliert Zugegriffen werden koennen.
  Mit jedem Segement sind Zugriffsattribute fuer jeden User der auf dieses Segment evtl. zugreifen will assoziiert.
  Diese Attribute werden von der Hardware ueberprueft. Jede Information kann direkt als Segment referenziert werden
  in anderen Systemen meist online informationen als Dateien referenziert werden. (unix, plan9 etc.)
  
  %S. 31
  In den meisten Systemen die Informationsteilung zur Verfuegung stellen sind die genannten beiden Kriterien nicht erfuellt.
  Als Beispiel wird im Paper \cite{inproc:multics} das CTSS System genannt, welches Informationen die geteilt werden sollen in Dateien vorhaelt.
  Das bedeutet das eine Kopie in den Buffer des Users kopiert werden muss damit der Benutzer darauf Zugreifen kann. (Benoetigt IO - Request)
  
  In nicht segmentierten Systemen das Benutzen von Kern Abbildern macht es sogut wie Unmoeglich den Zugriffskontrolle durchzusetzen.
  
 Die Hardware hat auf jedes Segment Zugriff mithilfe eines Segment descriptors, der die attribute beinhaltet.
 
 Multics gibt dem nutzer grosse Menge an segment descriptoren sodass Buffering der Informationen nicht mehr notwendig ist und die Informationen nicht
 ihre zugehoerigkeit verliert. Zudem heisst das ein Multics user keine Dateien benutzt sondern alle Informationen als Segmente referenziert., welche direkt vom Programm erreichbar sind.
 
 Fuer Multics users ersjcheint Speicher als eingeschlossen grosser unabhaengiger linearer Kernspeicher, jeder mit einem eigenen descriptor assoziiert. 
 Der Zugriff erfolgt durch [name, i] wobei name der symbolische Name des Segments und i ist die Wort Zahl im Linearen Speicher.
 Jeder User kann auf [name, i] referenzieren jedoch hat er nur die Zugriffsrechte die er auch vom Ersteller des Segments zugewiesen bekommen hat
 und die im Segment deskriptor eingetragen sind. Kombinationen von den folgenden Zugriffsrechten ''read'', ''write'', ''execute'' und ''append'' sind moeglich.
 
 Segmente koennen zudem in gleich grosse Teile geteilt werden sogenannten Pages. Dies wird auch in heutigen Systemen noch benutzt.
 Im Paper \cite{inproc:multics} wird auf die Vorteile drauf hingewiesen das die Allokierung durch die gleiche Groesse sehr vereinfacht wird,
 das die Pages nur im Speicher sein muessen die auch benoetigt werden das bedeutet das Segment groessen keine Einschraenkungen mehr besitzen.
 Sie weisen auch darauf hin das die Pages angefordert werden koennen die gebraucht werden und das keine anderen dann in den Speicher mit uebertragen werden muessen.
 Das ist auch eine Sache die heutzutage in aktuellen Betriebssystemen noch betrieben wird sogenanntes Paging welches mithilfe von
 verschieden Algorithmen wie FIFO (First In First Out), LRU (Last Recently used) oder anderen umgesetzt wird.
 
 Was heutzutage der TLB (translation lookaside buffer) bzw. die MMU macht (Seitennummern in Speicheradressen um mappen) hat in Multics der supervisor gemacht
 dieser transformierte den symbolischen Namen in eine passende Hardware Adresse, welche direkt vom Prozessor fuer weitere Referenzierungen benutzt wird.
 
 In der Multics Implementierung sind alle Segmente gepaged und die Page size betraegt 1024 Woerter.
 
 Prozesse sind Programme in der Ausfuehrung (definition \cite{inproc:multics}[S. 34]) Jeder Prozess in Multics besitzt ein privates Deskriptor Segement welches
 Segmentnummern in Kernspeicher adressen umwandelt und eine Private Tabelle welches die symbolischen Segmentnamen in Segmentnummern mapped. Wird
 auch als Known Segment Table KST bezeichnet. Supervisor in Multics ist im grunde wie der Kernel in heutigen Unix Systemen.
 Nicht wie in anderen Systemen arbeitet er in eigenem Speicherraum oder Prozess sondern die Supervisor Prozedur und deren Daten werden mit jedem Prozess geteilt. 
 Immer wenn ein neuer Prozess erstellt wird, wird das Deskriptor Segment mit den supervisor deskriptoren initialisiert. Dadurch muessen aber auch die Supervisor
 Segmente vor unautorisierten Zugriffen geschuetzt werden. Multics unterstuetzt den Benutzer mit einem sogenannten ring protection Mechanismen,
 welches die Segmente in dem Adressraum in einzelne Mengen mit verschiedenen Zugriffsrechten unterteilt. Dieser Mechanismus wird vom Supervisor zum
 eigenen Schutz verwendet.
 
 Name und Attribute eines Segments sind mit einem Katalog assoziiert. Im Grunde existiert eine Tabelle mit je einem Eintrag fuer ein Segment. 
 Ein Eintrag beinhaltet den Namen sowie die Segment Attribute. In Multics ist dieser Katalog als mehrere
 Segmente Implementiert und wird Verzeichnis bzw. directory genannt. Das Verzeichnis wird in einem Baum Strukturiert.
 D.h. ein Segmentname besteht aus einer Liste von einzelnen Subnamen die die Position des Eintrags im Baum reflektieren. (Subname = Entryname, liste = pathname)
 Pathname ist einzigartig in der ganzen Hierarchie und der Entryname nur im gegeben Directory.
  
  
%  \section{Capability systeme}
  
   
  \section{Kernel Unterschiede} \label{sec:kernel}
    Um sichere Betriebssysteme zu erstellen haben sich auch einige Wissenschaftler bzw. Informatiker mit der Verbesserung bzw. Verkleinerung
    des Kernels befasst. Sie haben diesen so strukturiert das dieser sicher ist und darauf aufbauend versucht ein sicheres Betriebssystem zu erzeugen.
  
\bibliographystyle{IEEEtran}
\bibliography{masterbib}
\end{document}
